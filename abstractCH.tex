\begin{abstractCH}

本篇論文討論如何利用近期所發展之深度學習技術在分散式表示在中文的資料集。 \\

近期深度學習受到極大的注目,相關技術也隨之蓬勃發展。然而相關的分散式表示方式,大多以英文為主的其他印歐語系作為主要的衡量對象,也據其特性發展。 
除了印歐語系外,另外漢藏語系及阿爾泰語系等也有眾多使用人口。 還有獨立語系的像日語、韓語等語系存在,各自也有其不同的特性。
中文本身屬於漢藏語系,本身具有相當不同的特性,像是孤立語、聲調、量詞等。
近來也有許多論文使用多語系的資料集作為評量標準,但鮮少去討論各語言間表現的差異。\\

本論文將利用前處理斷詞等方式,來比較近期所發展之深度學習之技術與傳統詞向量表示法的差異,也深入探討其模型對中文的展現。

\end{abstractCH}

