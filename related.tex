\chapter{Related Work}

In recent years, most models are aimed at English or more general way. 
\cite{zhao2012moodlens} performed the basic way to classify the articles from WeiBo with Naive Bayes.  \\

%There are also some researches about Emoticon in sentiment analysis like 
%\cite{Emojis}, which indicates the high correlation between emoticon and sentiment from the users. 

Additionally, sentiment analysis with typical deep learning models are conducted, like CNN 
\cite{kim2014convolutional}, RNN \cite{arevian2007recurrent}. 

When it comes to multiligual environment, the preprocess approach may differ in languages. Like Chinese and Japanese, segmentation may also invloved.
In the example of FastText\cite{joulin2016fasttext}, they also demostrated to convert character into pinyin, which make the subword infomation can be obtained. 

\cite{Dashtipour2016} summarized both corpus-base and lexicon-base techniques and list the languages those technique aimed at. 
Besides supervised methodology, there are some semi-supervised approaches.

There is also a work\cite{multilingual} to evaluate the multiligual approach and monoligual one. However, it used the Spanish and English as target, both two are belongs 
to Indo-European languages. It also addressed the culture difference, "dragon" mean harmful in English but it's opposite in Chinese. 
