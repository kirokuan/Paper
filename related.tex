\chapter{Related Work}


\section{Traditional Approach}

\cite{Dashtipour2016} summarized both corpus-base and lexicon-base techniques and listed the languages those techniques aimed at, and there are some innovative mothods combined both approaches. 
The advantage of corpus-base is that it is dictionary-free, but it requires relatively larger corpus to build the model, while lexicon-based approach depends mainly on existing resources to detect the sentiment.
When lexicon-based approach comes to the informal articles contributed by netizens, it may suffer some troubles like misspelling, abbreviation ,words or metaphor...etc, neither can it take the sequence of words into consideration.   
The basic corpus-based approach like TF-IDF is considered to be able to generate relatively good precision.
However, both approaches may utilized some keywords in the sentences rather than sentiment of sentence itself. In real world, we often use negation or irony to present our feeling rather than solely keywords. 
To solve the problems from rapidly-evolved languages, there are both semi-supervised and unsupervised approaches introduced as well. 

\section{Chinese Related Sentiment Analysis}

In recent years, most models are aimed at English or more general way. 

When it comes to multiligual environment, the preprocess approach may differ in languages. The traditional ways to counter the variation of words like stemming or lemmatization are appliable to most Latin languages.
Howerver, in Chinese and Japanese, segmentation may also be invloved. In the example of FastText\cite{joulin2016fasttext}, they also demostrated to convert character into pinyin, which make the subword infomation can be obtained. 

Though most approaches are tested and verified by English dataset, there are some work to test in Chinese dataset as well.

\cite{zhao2012moodlens} performed the basic way to classify the articles from WeiBo with Naive Bayes and smoothing with Laplace smooth.  
In this work, the authors also use emoticon as the ground truth to verify the approach, it also applied some imcrement learning. \\

%There are also some researches about Emoticon in sentiment analysis like 
%\cite{Emojis}, which indicates the high correlation between emoticon and sentiment from the users. 

\section{Advanced Approach}

Additionally, sentiment analysis with typical deep learning models are conducted, like CNN 
\cite{kim2014convolutional}, RNN \cite{arevian2007recurrent}, but most of them are applied in English dataset only. 

There is also a work\cite{multilingual} to evaluate the multiligual approach and monoligual one. However, it used the Spanish and English as target, both two are belongs 
to Indo-European languages. It also addressed the culture difference, "dragon" mean harmful in English but it's opposite in Chinese. 
