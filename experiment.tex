\chapter{Experiment}

\section{Set Up}
\begin{table}[]
\centering
\caption{Tag Category}
\label{CategoryTable}
\begin{tabular}{ll}
      &  \\
JOY  & 呵呵 酷 赞 贊 乐乐 鼓掌 耶 \\
DISGUST & 黑线 汗 晕 \\
SAD &   可憐 淚 衰 失望 伤心 泪 生病 囧 鄙视  淚 衰 失望 伤心 泪 生病 囧 鄙视  \\
FEAR &  委屈  可憐 \\
SURPRISE &  吃驚 吃惊 \\
ANGER & 怒 抓狂
\end{tabular}
\end{table}

The data set we chose is Open WeiboScope\cite{fu2013reality}, which is collected WeiBo randomly with API by researchers at the Journalism and Media Center of the University of Hong Kong in 2012. 
It contains 226 millions posts distributing over the year. We used the tags in posts as the indicators of sentiment,and removed some duplicated posts or some posts without any tags, or too many tags. 
We evaluated the accuracy of the classification for different algorithms.We used the TF-IDF and SVM (Joachims, 1998). as baseline.

For the data preprocessing and cleansing, it's a Weibo feature to allow the user to use emoticon, 
and the emoticon in raw data expressed as [笑](smile),[淚](tear). Like \cite{zhao2012moodlens},we also suffered the problem that the numbers of emoticon classes skewed,
 we deleted some posts from JOY and SAD randomly to make the dataset more balance.  

we removed the posts that contains too many tags, or without any tags. We also removed the duplicated posts by their post id roughly because it is a property of Chinese microblog \cite{fu2013reality} for Chinese netizens to post repeatedly, 
Besides, we only chose the post that over certain length (over above 10 characters).

The posts meets the criteria is about 7.4 millions. And we removed the tags in the original post, and there are so many tags 
,we use most-used 6 categories to categorize them as below.
