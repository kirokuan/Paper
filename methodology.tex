\chapter{Methods}

\begin{table}[]
\centering
\caption{Tag Category}
\label{CategoryTable}
\begin{tabular}{ll}
      &  \\
JOY  & 呵呵 酷 赞 贊 乐乐 鼓掌 耶 \\
DISGUST & 黑线 汗 晕 \\
SAD &   可憐 淚 衰 失望 伤心 泪 生病 囧 鄙视  可憐 淚 衰 失望 伤心 泪 生病 囧 鄙视  \\
FEAR &  委屈  可憐 \\
SURPRISE &  吃驚 吃惊 \\
ANGER & 怒 抓狂
\end{tabular}
\end{table}

The data set we chose is Open WeiboScope, which is collected WeiBo randomly by researchers at the Journalism and Media Center of the University of Hong Kong in 2012. It contains 226 millions posts distributing over the year. We used the tags in post as the indicators of sentiment,and removed some duplicated posts or some posts without any tags, or too many tags. We evaluated the accuracy of the classification for different algorithms.  We used the TF-IDF and SVM (Joachims, 1998). as baseline.

For the data preprocessing and cleansing, it's a Weibo feature to allow the user to use emoticon, and the emoticon in raw data expressed as [笑](smile),[淚](tear).

we removed the posts that contains too many tags, or without any tags. We also removed the duplicated posts by their post id roughly because it is a property of Chinese microblog \cite{fu2013reality} for Chinese netizens to post repeatedly, but most algorithms can't resist the duplicated posts. Besides, we only chose the post that over certain length .

The posts meets the criteria is about 7.4 millions. And we removed the tags in the original post, and there are so many tags , we use most-used 6 categories to categorize them as below.

We used jieba and dictionary to segment to post.



\section{The model introduction}

1. TF-IDF + SVM \\

	The conventional way to evaluate the semantics based on the occurrence of words and term, and it also takes the occurrence of word in global context into consideration.  It's simple and effective, but it still suffers from some disadvantages like data sparsity and high dimension. \\
	
2. FastText \cite{joulin2016fasttext}\\
	
	The structure of FastText is similar to CBOW of Mikolov et al. (2013), and it uses the softmax to compute the probabilities for predefined classes. The word representation is looked up through a table and finally averaged into the text representation. Finally it uses the linear classification.

3. Paragraph vector \cite{PVDB}\\

	This is method is purposed in \cite{PVDB}. The idea is obtain the summary of paragraphs,sentences or documents. There are 2 different algorithms we tested, which are dm (distributed memory) and dbow(distributed bag of words).

We use the implementation of Gensim and use SVM with linear kernel to classify.

4. Siamese-Cbow\cite{kenter2016siamesecbow}\\

	The method computes a sentence embedding is to average the embeddings of its
constituent words, instead of using pre-trained word embedding.

	We used the implementation (https://bitbucket.org/TomKenter/siamese-cbow/overview) from the author, and made it compatible with python3 for better compatibility with unicode.
