\chapter{緒論}
\setlength{\baselineskip}{1.5em}
\setlength{\parindent}{2em}
\setlength{\parskip}{1em}

\section{前言}

How to make the computer can operate the sentence with its own semantics more precisely is study of interest. Since the internet text volume grows so enormously and rapidly, how to make the information can be extracted more efficiently and precisely become more critical for many application. Chinese forums, blogs or microblog grow especially rapidly.    

Recently word2vec is considered to work for evaluated word semantics.  Additionally, the character is invariant to the language. Nevertheless, the problems in sentence level is more complicated, it's related to the sentence structure,  intention or context. There is several methods raised in recent years, like Siamese-CBOW, FastText ...etc. Most of them is able to train batch of text and construct the vectors.

\section{研究目的}

So far, most the studies are conducted in English, we are also interested if the feature also works in Chinese or other languages, and if the algorithm is invariant to the language grammar.