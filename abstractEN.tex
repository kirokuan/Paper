\begin{abstractEN}

The paper demonstrates how the deep learning methods published in recent years applied in Chinese sentence representation learning.

Recently, the deep learning techniques have attracted the great attention. Related areas also grow enormously. 
However, the most techniques use Indo-European languages mainly as evaluation objective, and developed corresponding to their properties.  
Besides Indo-European languages, there are Sino-Tibetan language and Altaic language, which also spoken widely. 
There are only some independent languages like Japanese or Korean, which have their own properties.
Chinese itself is belonged to Sino-Tibetan language family, and has some characters like isolating language, tone, count word...etc.
Recently, many publications also use the multilingual dataset to evaluate their performance, but few of them discuss the differences among different languages. 

This thesis demonstrates that we preform the sentiment analysis on Chinese Weibo dataset to quantize the effectiveness of different deep learning techniques.
We compared the traditional TF-IDF model with PVDM, Siamese-CBOW and fasttext, and evaluate the model they created.

\end{abstractEN}

