\begin{abstractEN}

The paper demonstrate how the deep learning methods published in recent years applied in Chinese dataset.

Recently, the deep learning attract the great attention. Related area also grow enormously. 
However, the most techniques use Indo-European languages mainly as evaluation objective, and develope corresponding to their properties.  
Besides Indo-European languages, there are Sino-Tibetan language and Altaic language, which also spoken widely. 
There are only some independant languages like Japanese or Korean, which have their own properties.
Chinese itself is belonged to Sino-Tibetan language family, and has some character like isolating language, tone, count word...etc.
Recently, many publication also use the multiligual dataset to evaluate their performance, but few of them discuess the difference brought by languages. 

The publication demostrates the difference from traditional techniques with the preproccess like segmentation, and dive into the models for Chinese. 

\end{abstractEN}

